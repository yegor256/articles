\documentclass{main}
\title{Using Independent Technical Reviews to Ensure Project Success}
\begin{document}
\begin{abstract}
Independent Technical Reviews can be used by business leaders and founders to
audit ongoing software development processes. Often enough, management lacks the
technical skills needed to understand and verify the work of internal and
external programmers, which creates risks. Management  depends upon software
developers to deliver the software that they will then market and sell. However,
if the dev team falls short or leaves a project, the management team could
quickly find themselves in a difficult spot.
\end{abstract}

Management understands the market, their client, and what type of software could
address pain points and drum up sales. Meanwhile, the dev team knows how to
code, but they may not understand the larger environment the software will be
used in. They may not understand the client’s pain points, wants, needs, and
abilities, for example. An independent technical reviewer can bridge the gap
between both parties by validating work and ensuring communication.

Many projects fail due to the rift between the dev team and management. And many
projects that ultimately succeed suffer a lot of bumps along the way, such as
going over budget or missing deadlines. Writing for Escrow London,
Evan Lever \href{https://www.escrowlondon.co.uk/news/5-reasons-software-development-projects-fail/}{notes}:
``The average overrun cost across all companies is 189\% of the original cost
estimate. In terms of time, the average overrun is 222\% of the original time
estimate. In order to alleviate the effect of overrun of time and cost,
expectations need to be carefully managed by the developer.''

If the development team falls short, or worse, abandons the project, the client
could be hung out to dry. Employers and investors  may find themselves with
little to show for their money besides indecipherable files of code. This
nightmare scenario happens often enough that it must be taken seriously.

You might try to bring in a new team, but it could be all but impossible for
them to suss out another team’s code. This is assuming that the code itself is
clean and conforms to industry standards and best practices. If not, the code
may be useless. One practice that is gaining popularity is to use an independent
technical reviewer, who can act as a third party technical expert.

By using independent technical reviews, management can utilize a third party to
audit the software and ensure that the dev team stays on track. Many independent
technical reviewers spend enough time working with business aspects that they
have learned to see things from a business leader’s perspective. This allows
them to become effective “middle men” between the dev and business teams,
enabling better communication and understanding.

Communication is vital. As industry leading blog
\href{https://medium.com/specstimate/10-reasons-why-software-development-projects-fail-7200e7c9ae2e}{Specstimate}
puts it:  ``Effective communication is
valuable in the workplace for so many reasons. It creates a healthy environment
for employees, helping them to work efficiently but it also creates a strong
relationship with clients and stakeholders.''

Gaps in knowledge and responsibility can lead to conflict and miscommunication.
In a worse case scenario, the dev team might abandon the software altogether due
to a lack of attention, oversight, and/or feedback. This is bad for everyone
involved, including the developers, who risk damaging their reputation.

Further, the independent technical reviewer introduces a third element to the
mix: someone who possesses technical skills but isn’t tied directly to the
development team, and someone who puts the business leader’s interests first and
foremost.

Independent technical reviewers are a boon for the development team as well. One
common reason dev teams abandon software is because they feel they aren’t
getting the needed direction or wanted attention for their efforts. This can
demotivate developers and result in abandonment.

An independent technical reviewer can provide feedback and attention,
encouraging the dev team onwards. As a technical expert, the reviewer can also
help the dev team discover and troubleshoot issues. Since the reviewer is
working closely with the business side, he or she will also be able to provide
direction and ensure that the project is on track.

An independent technical review should not be confused or replaced with an
internal code review. Yes, internal code reviews are useful and should be in
place. However, they lack the objectivity of an independent review and leave
business leaders exposed to needless risk.

Independent reviews can provide systemic, regular oversight for a project. While
internal reviews are often ad hoc and ongoing, independent reviews are better
organized and implemented at regular intervals. An independent reviewer can also
help uncover bugs, often much more quickly than the internal team, which may be
blind to them.

The result is a faster, better-organized development process that leads to
better outcomes for both the business and development teams. In a worst-case-
scenario, the independent reviewer can warn business leaders of serious issues,
or help recover their software should the dev team abandon it.

\end{document}
