\documentclass{main}
\title{Blockchain Was the First, But Will Not Remain the Only}
\begin{document}
\begin{abstract}
One of the most visible technical innovations of the last ten years,
aside from the promise to get tourists on Mars, is Blockchain. However,
due to the large amount of marketing noise and financial speculation, many
of us missed the technical trend this technology has successfully started.
The trend can be labeled Zero-Trust Decentralization (ZTD), and it
is truly unique and new. What is its future, and should we
expect other products aside from Blockchain?
\end{abstract}

Decentralized systems are not new at all. Take, for example, Torrent, a file sharing
protocol. It doesn't have a central place of control. Each node
works independently and anonymously, sending parts of the data to other nodes
and receiving other parts back.  However, despite our inability to trust
any individual node, the recipient of a file has a guarantee that it is
exactly the file that was initially injected into the network by its sender.
This guarantee is fairly easy to achieve just by checking hash-codes
of file parts and verifying them later at the time they are received.

Technically speaking, Torrent is a single-transaction system, where the
receiver and the sender trust each other. They only don't trust the system,
that's why they use checksums to make sure there was no data corruption
on the way.

A bigger problem arises when the receiver doesn't trust the system \emph{or}
the sender. A digital money sending business case is a perfect example. If money
is sent via, say, \href{https://www.paypal.com}{PayPal}, the receiver doesn't trust the sender, but it
does trust the system. The centralized database of PayPal Inc. guarantees
the seller of a book that the payment, with its unique ID, will
not disappear from the database after the book is shipped. But what if we also didn't trust
the PayPal system?

Is it possible to create an architecture where both the sender and the system
are anonymous and untrustworthy, but the combination of all participants
makes the entire system trustworthy to some extent? Blockchain was the first
solution which suggested that it was indeed possible.

In Blockchain, each node maintains the entire ledger of transactions
connected sequentially and signed with expensive hash codes. In order to modify
a transaction in the middle of the ledger a sender would have to recalculate
all the hash codes coming after it, which is a very time-consuming operation,
and all the more so as the tail of the ledger lengthens. On the other hand,
if such a recalculation is not done and the sender just changes the
transaction in the middle of the ledger, other nodes won't believe it and
will reject the modification. That's how their software is designed to behave
and they all have the same version of the software on-board.

This seems to be a solid concept, which proved itself in a number of real-life
applications, including Bitcoin and many other Blockchain-based cryptocurrencies.
However, this is just the beginning. This is the first, but not the only
possible solution in the territory of Zero-Trust Decentralization. There
are some, for example, based on Directed Acyclic Graphs, such as those used in
the IOTA and ByteBall cryptocurrencies.

\href{https://www.zold.io}{Zold}, yet another solution, which was invented and
developed by the author, shifts
the decision of trust to the client's side, allowing the money-receiving
software to decide which node has a more trustworthy version of the data. The data
coming from the majority of nodes will be considered as the most trustworthy.
A more detailed explanation of the architecture can be found in the
\href{https://papers.zold.io/wp.pdf}{White Paper}.

It is only reasonable to believe that there will be more attempts to solve
the same problem. They will use different principles of data organization,
different way to guarantee trust, and will have their own advantages and drawbacks. One of the
most visible disadvantages of Blockchain, for example, is its speed of transaction processing.
Obviously, since all transactions must be inserted
sequentially into the same ledger, it's impossible to achieve a high speed
of processing, unless some additional optimization is applied
(the Bitcoin Lightning Network and side-chains are examples).

Thus, the future of Zero-Trust Decentralization is indeed full of innovative
and interesting solutions. However, our current primary focus on Blockchain is
a potential threat to the further development of these innovations. We must
admit that Blockchain was a great concept which opened the door to the
new technical domain, but it's time to start researching and finding
better and more varied alternatives.

\end{document}
